\documentclass{jsarticle}
\usepackage{amsmath, amsthm, amssymb}
\usepackage{geometry}                % See geometry.pdf to learn the layout options. There are lots.
\geometry{a4paper}                   % ... or a4paper or a5paper or ... 
%\geometry{landscape}                % Activate for for rotated page geometry
%\usepackage[parfill]{parskip}    % Activate to begin paragraphs with an empty line rather than an indent
\usepackage{graphicx}
\usepackage{epstopdf}
\DeclareGraphicsRule{.tif}{png}{.png}{`convert #1 `dirname #1`/`basename #1 .tif`.png}


\input{/Users/jun/Dropbox/settings/my_macros.tex}

\title{仮説検定}
\author{2018年 前期講義資料 (作成者:大塚 淳)}
\date{ }                                           % Activate to display a given date or no date

\begin{document}
\maketitle

\section{Neyman-Pearsonの仮説検定}
ネイマン・ピアソンの仮説検定では、帰無仮説$H_0$と対立仮説$H_1$があるとき、与えられたデータをもとに、帰無仮説を棄却する(対立仮説を採択する)ことができるかどうかを判定する。
ここでのポイントは、仮説が蓋然的である限り、そうした判定は常に誤る可能性があるということである。
NPはこれを受け入れたうえで、データをもとに検定を行ったとき、それが平均してどれだけ誤るのかを計算し、それをもとに仮説の棄却・採択を行う。

授業でも述べるように、ここには二種類の誤りの可能性がある:
\begin{quote}
 \begin{description}
 \item[第1種の誤り] $H_0$が真であるのに、それを誤って棄却してしまう(偽陽性)
 \item[第2種の誤り] $H_0$が偽であるのに、それを棄却しそこなう(偽陰性)
 \end{description}
\end{quote}
どのようなテストにも、この二つの誤りの可能性はつきまとう。
テストが第1種の誤りを犯す確率を$\alpha$、第2種の誤りを犯す確率を$\beta$とする。
後で見るように、多くの場合この二つはトレードオフの関係にある。
そこでNPはまず$\alpha$を低い値に定め、その基準を満たすテストの中から、もっとも低い$\beta$を持つテスト選ぶ、というようにしてこの問題を解決する。
以下では、簡単な例にもとづいて、検定のアイデアと、$\alpha, \beta$の計算方法を示す。

\section{有意水準と第1種の誤り}
ある国で発行されているコインは、淵が丸まっているので、立てようとしてもどちらかに倒れてしまう。
肉眼では見分けはつかないが、実はこの丸み処理には2種類あって、最近発行されたものは1/4の確率で表を上にして倒れるが、旧式のものは3/4で表がでる。
旧式のものは今やレアで、コレクター市場ではかなりの値が付くそうだ。
さて、あなたの叔父さんがこの国に旅行して、おみやげとしてこの旧式コインをくれた。
しかしあなたはどうも確信がもてない。これは本当にレアな旧式コインなのか?

そこであなたは、そのコインを10回立てる実験を行って判断することにした。
実は旧式ではないという帰無仮説は$H_0: p=0.25$、正真正銘のレアものであるという対立仮説は$H_0: p=0.75$である。
もし$H_0$が正しかったら、10回のうち何回表がでると期待できるだろうか?
表の回数を$X$としたとき、$P(X=x)$の確率は、$p=0.25, n=10$の二項分布から次のように求まる。
\[
  P(X=x) = {}_{10} \mathrm{C}_{x}  (0.25)^{x} (0.75)^{10-x}
\]
\begin{figure}[ht]
 \centering
 \includegraphics[width=8cm]{r/binom_p025_n10.eps}
 \caption{$H_0$のもとでの$P(X=x)$}
\end{figure}

図から明らかなように、帰無仮説は右に行くほど(表が多いほど)起こりにくいと予想する。
そこで、このように予測から外れる結果が実際に出た際に帰無仮説を棄却する、という方針が考えられる。
では、何回以上表が出たら「予測が外れた」と判断すべきだろうか?
同じ分布から:
\begin{enumerate}
 % \item $P(X=20) = 0.00000000000091 \sim$ 10兆回に1回
 % \item $P(X \geq 10) \sim 0.0139 \sim$ 100回に1回強 
 % \item $P(X \geq 9) \sim 0.0409 \sim$ 20回に1回弱 
 \item $P(X=10) = 0.00000095$
 \item $P(X \geq 6) \sim 0.020$
 \item $P(X \geq 5) \sim 0.078$
\end{enumerate}

これをもとに、3つのテストを考えてみる
\begin{itemize}
 \item[i. ] いま、1のときだけ、つまり全部が表だったときだけ$H_0$を棄却すると決めたとしよう。
       このとき実は$H_0$が真なのに棄却されてしまう、つまりこのテストが第1種の誤りを犯すのは、せいぜい100万回に1回ということになる。
 \item[ii. ] 次に、2のとき、つまり6回以上表だったら$H_0$を棄却するテストを考えると、これが誤って$H_0$を棄却する確率は2\%弱である。
 \item[iii. ] 最後に、半分以上表が出たら$H_0$を棄却するテストを考えると、3よりこの第1種の誤りの確率は8\%弱となる
\end{itemize}

iのテストは極めて慎重だが、ほとんど帰無仮説を棄却できず感度が悪い。
一方iiiを用いれば、比較的わずかなズレでも$H_0$を棄却できるが、しかし誤って棄却してしまうリスクもそれなりにある。
よって上の1,2,3の確率は、i, ii, iiiのテストによる帰無仮説の棄却しやすさと同時に、それが正しい時に誤って棄却してしまう第1種の誤りの確率$\alpha$を与える。
これを\emph{有意水準} (significance level)といい、有意水準以下になるような(つまり「予測から外れた」)データの範囲を\emph{棄却域}(critical region)という。
%帰無仮説の分布が与えられていれば、有意水準が決まれば棄却域は自動的に決まる。
仮説検定ではまず有意水準を定め、実際の結果がその棄却域に入っていたら帰無仮説を棄却する。
有意水準の設定は分野や文脈によって異なるが、慣習的に5\%とされることが多い。
ここでこの基準を採用するなら、コインが6回以上表がでたとき、$H_0$を棄却すれば良いことになる。

$\alpha$が第1種の誤りの確率を示しているとすれば、$1-\alpha$はテストの厳しさ、つまり$H_1$が偽であるときそれをどれくらいの確率で見抜けるか、を表している。これをテストの\emph{サイズ}という。
サイズが高いテストほど(有意水準が低いほど)、$H_0$の棄却/$H_1$の採択が偶然の産物である可能性は低くなるわけで、その意味でその結果はより有意である(significant)といえる。


\section{検出力と第2種の誤り}
次に、第2種の誤り、つまり対立仮説$H_1$が真であるときに、帰無仮説$H_0$を棄却しそこねる確率を考えてみる。
$H_1$が真であるとは、つまり$P(X=x)$は本当は$p=0.75, n=10$の二項分布に従う、つまり真実は図1でなく以下の図2のようである、ということだ。
\[
  P(X=x) = {}_{10} \mathrm{C}_{x}  (0.75)^{x} (0.25)^{10-x}
\]
\begin{figure}[h]
 \centering
 \includegraphics[width=8cm]{r/binom_p075_n10.eps}
 \caption{$H_1$のもとでの$P(X=x)$}
\end{figure}

もちろん我々には、図1と2のどちらが真実なのかはわからず(わかっていたら検定などする必要がない)、粛々と上で選んだテストを行うだけである。
さて、そのテストを行ったところ、帰無仮説を棄却できなかったとしよう。
このとき気になるのは、この結論がどのくらい信頼のおけるものなのか、ということだ。
つまり、実は$H_0$は偽($H_1$が真)なのにそれを見過ごしてしまっている確率、テストが第2種の誤りを犯す確率を知りたい。
これは図2の分布をもとに、以下のように計算できる。
\begin{enumerate}
 \item[1'.] $P(X<10) = 0.944 \sim$ 100回に95回強 
 \item[2'.] $P(X < 6) = 0.078 \sim$ 100回に8回弱 
 \item[3'.] $P(X < 5) = 0.020 \sim$ 100回に2回 
\end{enumerate}

これがそれぞれ上で考えたi, ii, iiiのテストが第2種の誤りを犯す確率$\beta$となる。その理由を考えてみよう。
その際、以下がヒントとなる。
\begin{itemize}
 \item 第2種の誤り($H_0$が偽/$H_1$が真であるときに$H_0$を棄却しない)を問題にしているのだから、
       真実として想定すべき確率分布は図\underline{\hspace{1cm}}である
 \item i, ii, iiiのテストが$H_0$を棄却しないのはそれぞれ\underline{\hspace{2cm}}, \underline{\hspace{2cm}}, \underline{\hspace{2cm}}のときである
 \item よって第2種の誤りの確率はそれぞれ\underline{\hspace{2cm}}, \underline{\hspace{2cm}}, \underline{\hspace{2cm}}である
\end{itemize}

第2種の誤りとは、本当は$H_1$が正しいのに、それを見逃してしまう、ということだ(偽陰性)。
よってその確率$\beta$が低いほど、感度の高いテストだと言える。
ここから、$1-\beta$を、そのテストの\emph{検出力}(power)という。
通常、我々の興味ある仮説が対立仮説として設定されるので、検出力はテストが興味ある結果をどれだけしっかり検知できるか、ということを表している。

このように、Neyman-Pearsonの検定理論では蓋然的な仮説をテストするに当たり2種類の誤りを考慮する。
どちらをより重視すべきか?
NPでは、このうち第1種の誤りの方を重視する。つまり、帰無仮説を棄却するには、それを採択するよりもより慎重であるべきとされる。
よって検定では、まず有意水準$\alpha$を定め、その基準をクリアするテストのうち最も$\beta$が低いような(検出力が高いような)ものを選ぶ、というのが一般的な戦略である。

\section{仮説検定まとめ}

\paragraph{ポイント}
\begin{itemize}
 \item テストは2つの指標、$\alpha$と$\beta$、ないしはサイズ$(1-\alpha)$と検出力$(1-\beta)$で特徴づけられる
       \begin{itemize}
	\item $\alpha$(有意水準)は$H_0$が真のときそれを誤って退ける第1種の誤りの確率を表す
	\item $\beta$は$H_1$が真のときそれを見逃す第2種の誤りの確率を表す
       \end{itemize}
 \item NPでは、一定の$\alpha$の基準を満たすテストのうち、$\beta$を最小にする(検出力を最大にする)ようなテストが推奨される
\end{itemize}

\paragraph{検定のステップ}
\begin{enumerate}
 \item 帰無仮説と対立仮説を決める $\rightarrow$ 両仮説のもとでの分布がそれぞれ定まる
 \item どれくらいの「ありえなさ」だったら帰無仮説を棄却するか(有意水準)を定める \\
       $\rightarrow$ これが同時にテストの第1種の誤りの確率$\alpha$となる
 \item 有意水準と帰無仮説$H_0$のもとでの分布から、棄却域が定まる
 \item 実験を行い、結果が棄却域内であれば、帰無仮説を棄却する
 \item 棄却できなかった場合、対立仮説$H_1$のもとでの分布をもとに第2種の誤りの確率を求める
\end{enumerate}


\section{サンプルサイズの重要性}
ベイズ推論のところで我々は、観測を重ねるほど結論が精確になっていく過程を見た(swamping of priors)。
仮説検定においても、サンプルサイズは重要な役割を持つ。
しかしその意味合いは異なる。
ベイズ推論では大サンプルは\kenten{結論}の精確さに寄与するのに対し、検定では\kenten{テスト}の精確さに関わる。
これを見るために、上と同じ状況だが、コインの試行回数を10でなく20回に増やした実験を考えてみよう。

$n=20$としたときの$H_0, H_0$のもとでのコインの表がでる回数$X$の確率分布は以下のとおり
\begin{figure}[h]
 \centering
 \includegraphics[width=8cm]{r/double_binom_n20.eps}
 \caption{$n=20$としたときの$P(X=x)$。ただし見やすいようにプロットを曲線化し、$H_0$のもとでの分布を実線、$H_1$を点線で表している。}
\end{figure}

ここで有意水準を5\%とするテストを考えると、帰無仮説$H_0$のもとでの棄却域は$P(X \geq 9)=0.041$より、そのテストは表が9回以上でたとき、$H_0$を棄却することになる。つまり、表がでるのが半分以下でも帰無仮説を棄却できるという意味で、$n=10$のときの同水準のテストよりも感度が高くなっている。

また、そのテストが仮説を棄却できなかった場合、つまり表が8回未満しかでなかった場合を考えよう。
対立仮説$H_1$のもとで、$P(X < 9)= 0.00094$。
したがって仮に帰無仮説が偽であったときに、そのテストがそれを見逃してしまう確率は1000回に1回程度である。

我々は図1と図2で、テストの感度(サイズ)と検出力がトレードオフの関係にあることを見た。
棄却域を右にずらすほど、テストの感度・サイズ(図1で棄却域の左側にくる確率)は上がるが、検出力(図2で棄却域の右側にくる確率)は下がる。
しかしサンプルサイズを増やすことができれば、$H_0, H_1$の分布の分散を減らすことで、テストの感度と検出力を共に上げることができる。
こうした意味で、検定理論においても沢山のデータを集めることはより確実な推論を可能にするのである。



\section*{練習問題}
\begin{enumerate}
 \item 2節で述べた実験について、有意水準を10\%としたときの棄却域を図1より求めよ。またそのときの検出力を求めよ。
 \item 上のテストで、めでたく帰無仮説を棄却できたとしよう。そのとき、この結果が意味することとして正しいものをすべて選べ
       \begin{enumerate}
	\item これが旧式のレアコインである確率は10\%以下だ
	\item これが旧式のレアコインである確率は90\%以上だ
	\item もしこれがレアコインでなかったとしたら、こんな結果は10回に1回もでないだろう
	\item もしこれがレアコインでなかったとしても、10回に9回はこのような結果がでただろう
       \end{enumerate}
 \item ベイズ主義者だったら、叔父さんがくれたコインがレアかどうかを決めるためにどうするだろうか。また、そのときに必要となる情報はなんだろうか。
\end{enumerate}




\end{document}